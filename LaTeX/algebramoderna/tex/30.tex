\section*{Sección 30}

\begin{Teorema}
	Sea $E$ una extensión finita de F, y sea $\alpha \in E$ algebraica sobre F. si el $deg(\alpha,F)=n$, Entonces $F(\alpha)$ es un espacio vectorial n-dimensional sobre $F$ con basico $\{1,\alpha, \cdots , \alpha^{n-1}\}$ sin embargo, cada elemento $\beta$ de $F(\alpha)$ es algebraica sobre F, y $deg(\beta,F) \leq deg(\alpha,F)$
	\vspace*{0.3cm}
\end{Teorema}

En los Ejercicios del 4 al 9, da una base para el espacio vectorial indicado sobre el campo:

\begin{enumerate}
\setcounter{enumi}{3}
    \item \( \mathbb{Q}(\sqrt{2}) \) sobre \( \mathbb{Q} \)
    
    \textbf{Solución:} Como \( \sqrt{2} \) es una raíz del irreducible \( x^2 - 2 \) de grado 2, el Teorema 30.23 muestra que una base es \(\{1, \sqrt{2}\}\).

    \item \( \mathbb{R}(\sqrt{2}) \) sobre \( \mathbb{R} \)

    \textbf{Solución:} Dado que \( \sqrt{2} \) está en \( \mathbb{R} \) y es una raíz del polinomio \(x - \sqrt{2}\) de grado 1, el Teorema 30.23 muestra que una base es \(\{1\}\).

    \item \( \mathbb{Q}(\sqrt[3]{2}) \) sobre \( \mathbb{Q} \)

    \textbf{Solución:} Como \( \sqrt[3]{2} \) es una raíz del irreducible \(x^3 - 2\) de grado 3, según el Teorema 30.23 una base es \(\{1, \sqrt[3]{2}, (\sqrt[3]{2})^2\}\).

    \item \( \mathbb{C} \) sobre \( \mathbb{R} \)

    \textbf{Solución:} Dado que \( \mathbb{C} = \mathbb{R}(i) \) donde \( i \) es una raíz del irreducible \(x^2 + 1\) de grado 2, el Teorema 30.23 muestra que una base es \(\{1, i\}\).

    \item \( \mathbb{Q}(i) \) sobre \( \mathbb{Q} \)

    \textbf{Solución:} Dado que \( i \) es una raíz del irreducible \(x^2 + 1\) de grado 2, el Teorema 30.23 muestra que una base es \(\{1, i\}\).

    \item \( \mathbb{Q}(\sqrt[4]{2}) \) sobre \( \mathbb{Q} \)

    \textbf{Solución:} Dado que \( \sqrt[4]{2} \) es una raíz del irreducible \(x^4 - 2\) de grado 4, según el Teorema 30.23 una base es \(\{1, \sqrt[4]{2}, \sqrt{2}, (\sqrt[4]{2})^3\}\).

\end{enumerate}

\section*{Sección 31}

\subsection*{Calculos}
En los Ejercicios 1 a 13, encuentra el grado y una base para la extensión de campo dada. Prepárate para justificar tus respuestas.

\begin{enumerate}
    \item $\mathbb{Q}(\sqrt{2})$ sobre $\mathbb{Q}$
    
    \textbf{Solución:} Como \( \sqrt{2} \) es una raíz del irreducible \( x^2 - 2 \), el grado es 2 y una base es \(\{1, \sqrt{2}\}\).

    \item \( \mathbb{Q}(\sqrt{2}, \sqrt{3}) \) sobre \( \mathbb{Q} \)

    \textbf{Solución:} Por el Ejemplo 31.9, el grado es 4 y una base es \(\{1, \sqrt{2}, \sqrt{3}, \sqrt{6}\}\).

    \item \( \mathbb{Q}(\sqrt{2}, \sqrt{3}, \sqrt{18}) \) sobre \( \mathbb{Q} \)

    \textbf{Solución:} Observamos que \( \sqrt{18} = \sqrt{2} \cdot \sqrt{3} \sqrt{3} \). Por lo tanto, \( \mathbb{Q}(\sqrt{2}, \sqrt{3}, \sqrt{18}) \) y \( \mathbb{Q}(\sqrt{2}, \sqrt{3}) \) son el mismo campo. El grado es 4 y una base es \(\{1, \sqrt{2}, \sqrt{3}, \sqrt{6}\}\) según el Ejemplo 31.9.

    \item \( \mathbb{Q}(\sqrt[3]{2}, \sqrt{3}) \) sobre \( \mathbb{Q} \)

    \textbf{Solución:} Dado que \( \sqrt{3} \notin \mathbb{Q}(\sqrt[3]{2}) \) porque \( \mathbb{Q}(\sqrt{3}) \) tiene grado 2 sobre \( \mathbb{Q} \) mientras que \( \mathbb{Q}(\sqrt[3]{2}) \) tiene grado 3, y 2 no divide a 3, el grado de \( \mathbb{Q}(\sqrt[3]{2}, \sqrt{3}) \) sobre \( \mathbb{Q} \) es 6. Formamos productos a partir de las bases \(\{1, \sqrt{3}\}\) para \( \mathbb{Q}(\sqrt{3}) \) sobre \( \mathbb{Q} \) y \(\{1, \sqrt[3]{2}, (\sqrt[3]{2})^2\}\) para \( \mathbb{Q}(\sqrt[3]{2}) \) sobre \( \mathbb{Q}(\sqrt{3})\), obteniendo \(\{1, \sqrt[3]{2}, (\sqrt[3]{2})^2, \sqrt{3}, \sqrt{3}\sqrt[3]{2}, \sqrt{3}(\sqrt[3]{2})^2\}\) como una base.

    \item \( \mathbb{Q}(\sqrt{2}, \sqrt[3]{2}) \) sobre \( \mathbb{Q} \)

    \textbf{Solución:} Como en la solución al Ejercicio 4, la extensión tiene grado 6. Tomando productos de las bases \(\{1, \sqrt{2}\}\) para \( \mathbb{Q}(\sqrt{2}) \) sobre \( \mathbb{Q} \) y \(\{1, \sqrt[3]{2}, (\sqrt[3]{2})^2\}\) para \( \mathbb{Q}(\sqrt[3]{2}) \) sobre \( \mathbb{Q}(\sqrt{2})\), vemos que \(\{1, \sqrt[3]{2}, (\sqrt[3]{2})^2, \sqrt{2}, \sqrt{2}\sqrt[3]{2}, \sqrt{2}(\sqrt[3]{2})^2\}\) es una base. Es fácil ver que \( \mathbb{Q}(\sqrt{2}, \sqrt[3]{2}) = \mathbb{Q}(\sqrt[6]{2}) \) ya que \( 2^{1/6} = (2^{1/3})^{1/2} \), así que otra base es \(\{1, 2^{1/6}, (2^{1/6})^2, (2^{1/6})^3, (2^{1/6})^4, (2^{1/6})^5\}\).

    \item \( \mathbb{Q}(\sqrt{2} + \sqrt{3}) \) sobre \( \mathbb{Q} \)

    \textbf{Solución:} Como se muestra en el Ejemplo 31.9, tenemos grado 4, entonces \( \mathbb{Q}(\sqrt{2} + \sqrt{3}) = \mathbb{Q}(\sqrt{2}, \sqrt{3}) \) y una base es \(\{1, \sqrt{2}, \sqrt{3}, \sqrt{6}\}\) tal como en el Ejemplo 31.9.

    \item \( \mathbb{Q}(\sqrt{2}\sqrt{3}) \) sobre \( \mathbb{Q} \)

    \textbf{Solución:} Porque \( \sqrt{2}\sqrt{3} = \sqrt{6} \), vemos que el campo es \( \mathbb{Q}(\sqrt{6}) \) que tiene grado 2 sobre \( \mathbb{Q} \) y una base es \(\{1, \sqrt{6}\}\).

    \item \( \mathbb{Q}(\sqrt{2}, \sqrt[3]{5}) \) sobre \( \mathbb{Q} \)

    \textbf{Solución:} Como en la solución al Ejercicio 4, vemos que la extensión es de grado 6. Formamos productos de las bases \(\{1, \sqrt{2}\}\) para \( \mathbb{Q}(\sqrt{2}) \) sobre \( \mathbb{Q} \) y \(\{1, \sqrt[3]{5}, (\sqrt[3]{5})^2\}\) para \( \mathbb{Q}(\sqrt[3]{5}) \) sobre \( \mathbb{Q}(\sqrt{2})\), obteniendo \(\{1, \sqrt[3]{5}, (\sqrt[3]{5})^2, \sqrt{2}, \sqrt{2}\sqrt[3]{5}, \sqrt{2}(\sqrt[3]{5})^2\}\) como una base.

    \item \( \mathbb{Q}(\sqrt[3]{2}, \sqrt[3]{6}, \sqrt[3]{24}) \) sobre \( \mathbb{Q} \)

    \textbf{Solución:} Ahora, \( \frac{\sqrt[3]{6}}{\sqrt[3]{2}} = \sqrt[3]{3} \) y \( \sqrt[3]{24} = 2\sqrt[3]{3} \), entonces \( \mathbb{Q}(\sqrt[3]{2}, \sqrt[3]{6}, \sqrt[3]{24}) = \mathbb{Q}(\sqrt[3]{2}, \sqrt[3]{3}) \). El grado sobre \( \mathbb{Q} \) es 9, y tomamos productos de las bases \(\{1, \sqrt[3]{2}, (\sqrt[3]{2})^2\}\) y \(\{1, \sqrt[3]{3}, (\sqrt[3]{3})^2\}\) para \( \mathbb{Q}(\sqrt[3]{2}) \) sobre \( \mathbb{Q} \) y \( \mathbb{Q}(\sqrt[3]{2}, \sqrt[3]{3}) \) sobre \( \mathbb{Q}(\sqrt[3]{2}) \) respectivamente, obteniendo la base \(\{1, \sqrt[3]{2}, \sqrt[3]{4}, \sqrt[3]{3}, \sqrt[3]{6}, \sqrt[3]{12}, \sqrt[3]{9}, \sqrt[3]{18}, \sqrt[3]{36}\}\).

    \item \( \mathbb{Q}(\sqrt{2}, \sqrt{6}) \) sobre \( \mathbb{Q}(\sqrt{3}) \)

    \textbf{Solución:} Dado que \( \mathbb{Q}(\sqrt{2}, \sqrt{6}) = \mathbb{Q}(\sqrt{2}, \sqrt{3}) \), la extensión tiene grado 2 sobre \( \mathbb{Q}(\sqrt{3}) \) y tomamos el conjunto \(\{1, \sqrt{2}\}\) como una base.

    \item \( \mathbb{Q}(\sqrt{2} + \sqrt{3}) \) sobre \( \mathbb{Q}(\sqrt{3}) \)

    \textbf{Solución:} Por el Ejemplo 31.9, \( \mathbb{Q}(\sqrt{2} + \sqrt{3}) = \mathbb{Q}(\sqrt{2}, \sqrt{3}) \), entonces el grado de la extensión es 2 y tomamos el conjunto \(\{1, \sqrt{2}\}\) como una base sobre \( \mathbb{Q}(\sqrt{3}) \).

    \item \( \mathbb{Q}(\sqrt{2}, \sqrt{3}) \) sobre \( \mathbb{Q}(\sqrt{2} + \sqrt{3}) \)

    \textbf{Solución:} Por el Ejemplo 31.9, \( \mathbb{Q}(\sqrt{2}, \sqrt{3}) = \mathbb{Q}(\sqrt{2} + \sqrt{3}) \), entonces el grado de la extensión es 1 y tomamos el conjunto \(\{1\}\) como una base sobre \( \mathbb{Q}(\sqrt{2} + \sqrt{3}) \).

    \item \( \mathbb{Q}(\sqrt{2}, \sqrt{6} + \sqrt{10}) \) sobre \( \mathbb{Q}(\sqrt{3} + \sqrt{5}) \)

    \textbf{Solución:} Ahora, \( \sqrt{6} + \sqrt{10} = \sqrt{2}(\sqrt{3} + \sqrt{5}) \), entonces \( \mathbb{Q}(\sqrt{2}, \sqrt{6} + \sqrt{10}) = \mathbb{Q}(\sqrt{2}, \sqrt{3} + \sqrt{5}) \). El grado de la extensión es 2 y una base sobre \( \mathbb{Q}(\sqrt{3} + \sqrt{5}) \) es \(\{1, \sqrt{2}\}\).

\end{enumerate}

\subsection*{Teoría}

\begin{enumerate}
\setcounter{enumi}{21}
    \item Demuestra que si $(a + bi)$ pertenece a $\mathbb{C}$ donde $a, b$ pertenecen a $\mathbb{R}$ y $b \neq 0$, entonces $\mathbb{C} = \mathbb{R}(a + bi)$.
    
    \textbf{Solución:}
    
    Si $b \neq 0$, entonces $a + bi$ es un número complejo donde $a, b$ son números reales. Por el Teorema 31.3, $a + bi$ es algebraico sobre $\mathbb{R}$. Luego, por el Teorema 31.4,
    \[ [\mathbb{C} : \mathbb{R}] = [\mathbb{C} : \mathbb{R}(a+bi)][\mathbb{R}(a+bi) : \mathbb{R}] = 2. \]
    Dado que $a+bi \notin \mathbb{R}$, debemos tener $[\mathbb{R}(a+bi) : \mathbb{R}] = 2$, por lo tanto $[\mathbb{C} : \mathbb{R}(a+bi)] = 1$. Así, $\mathbb{C} = \mathbb{R}(a+bi)$.
    
    \item Muestra que si $E$ es una extensión finita de un campo $F$ y $[E : F]$ es un número primo, entonces $E$ es una extensión simple de $F$ y, de hecho, $E = F(a)$ para cada $a$ en $E$ que no está en $F$.
    
    \textbf{Solución:}
    
    Sea $\alpha$ cualquier elemento en $E$ que no esté en $F$. Entonces, $[E:F] = [E:F(\alpha)][F(\alpha):F] = p$ para algún primo $p$ según el Teorema 31.4. Dado que $\alpha$ no está en $F$, sabemos que $[F(\alpha) : F] > 1$, por lo que debemos tener $[F(\alpha) : F] = p$ y, por lo tanto, $[E : F(\alpha)] = 1$. Esto muestra que $E = F(\alpha)$, que es lo que deseamos demostrar.
    
    \item Demuestra que $x^3 - 3$ es irreducible sobre $\mathbb{Q}(\sqrt[3]{2})$. 
    
    \textbf{Solución:}
    
    Si $x^2 - 3$ fuera reducible sobre $\mathbb{Q}(\sqrt[3]{2})$, entonces se factorizaría en factores lineales sobre $\mathbb{Q}(\sqrt[3]{2})$, por lo que $\sqrt{3}$ estaría en el campo $\mathbb{Q}(\sqrt[3]{2})$, y tendríamos $\mathbb{Q}(\sqrt{3}) \subseteq \mathbb{Q}(\sqrt[3]{2})$. Pero entonces, por el Teorema 31.4,
    \[ [\mathbb{Q}(\sqrt[3]{2}) : \mathbb{Q}] = [\mathbb{Q}(\sqrt[3]{2}) : \mathbb{Q}(\sqrt{3})][\mathbb{Q}(\sqrt{3}) : \mathbb{Q}]. \]
    Esta ecuación es imposible porque $[\mathbb{Q}(\sqrt[3]{2}) : \mathbb{Q}] = 3$ mientras que $[\mathbb{Q}(\sqrt{3}) : \mathbb{Q}] = 2$.
\end{enumerate}



\begin{enumerate}
    \setcounter{enumi}{25}
    
    \item Sea $E$ una extensión de campo finita de $F$. Sea $D$ un dominio integral tal que $F \subseteq D \subseteq E$. Demuestra que $D$ es un campo.
    
    \textbf{Solución:}
    
    Solo necesitamos demostrar que para cada $\alpha \in D$ con $\alpha \neq 0$, su inverso multiplicativo $1/\alpha$ también está en $D$. Como $E$ es una extensión finita de $F$, sabemos que $\alpha$ es algebraico sobre $F$. Si $\text{deg}(\alpha, F) = n$, entonces por el Teorema 30.23, tenemos:
    
    \[ F(\alpha) = \left\{ a_0 + a_1 \alpha + a_2 \alpha^2 + \cdots + a_{n-1} \alpha^{n-1} \mid a_i \in F \text{ para } i = 0, \dots, n-1 \right\}. \]
    
    En particular, $1/\alpha \in F(\alpha)$, por lo que $1/\alpha$ es un polinomio en $\alpha$ con coeficientes en $F$, y por lo tanto está en $D$.
    
    \item Demuestra en detalle que $\mathbb{Q}(\sqrt{3} + \sqrt{7}) = \mathbb{Q}(\sqrt{3}, \sqrt{7})$.
    
    \textbf{Solución:}
    
    Es obvio que $\mathbb{Q}(\sqrt{3} + \sqrt{7}) \subseteq \mathbb{Q}(\sqrt{3}, \sqrt{7})$. Ahora, $(\sqrt{3} + \sqrt{7})^2 = 10 + 2 \sqrt{21}$, por lo que $\sqrt{21} \in \mathbb{Q}(\sqrt{3} + \sqrt{7})$. Por lo tanto,
    
    \[ (\sqrt{3} + \sqrt{7}) - \sqrt{7} = \sqrt{3} \]
    
    también está en $\mathbb{Q}(\sqrt{3} + \sqrt{7})$. De manera similar, $\sqrt{3} + \sqrt{7} - \sqrt{3} = \sqrt{7}$, por lo que $\mathbb{Q}(\sqrt{3}, \sqrt{7}) \subseteq \mathbb{Q}(\sqrt{3} + \sqrt{7})$. Por lo tanto, $\mathbb{Q}(\sqrt{3}, \sqrt{7}) = \mathbb{Q}(\sqrt{3} + \sqrt{7})$.
    
    \item Generalizando el Ejercicio 27, demuestra que si $\sqrt{a} + \sqrt{b} \neq 0$, entonces $\mathbb{Q}(\sqrt{a} + \sqrt{b}) = \mathbb{Q}(\sqrt{a}, \sqrt{b})$ para todo $a$ y $b$ en $\mathbb{Q}$. [Pista: Calcula $\frac{a - b}{\sqrt{a} + \sqrt{b}}$.]
    
    \textbf{Solución:}
    
    Si $a = b$, el resultado es claro; asumimos entonces que $a \neq b$. Es evidente que $\mathbb{Q}(\sqrt{a} + \sqrt{b}) \subseteq \mathbb{Q}(\sqrt{a}, \sqrt{b})$. Ahora mostraremos que $\mathbb{Q}(\sqrt{a}, \sqrt{b}) \subseteq \mathbb{Q}(\sqrt{a} + \sqrt{b})$. Sea $\alpha = \frac{a - b}{\sqrt{a} + \sqrt{b}} \in \mathbb{Q}(\sqrt{a} + \sqrt{b})$. Entonces $\alpha = \sqrt{a} - \sqrt{b}$. Por lo tanto, $\mathbb{Q}(\sqrt{a} + \sqrt{b})$ contiene $\frac{1}{2}[\alpha + (\sqrt{a} + \sqrt{b})] = \frac{1}{2} (2\sqrt{a}) = \sqrt{a}$ y por lo tanto también contiene $(\sqrt{a} + \sqrt{b}) - \sqrt{a} = \sqrt{b}$. Así que $\mathbb{Q}(\sqrt{a}, \sqrt{b}) \subseteq \mathbb{Q}(\sqrt{a} + \sqrt{b})$.
    
    \item Sea $E$ una extensión finita de un campo $F$, y sea $p(x)$ en $F[x]$ irreducible sobre $F$ y tenga grado que no sea un divisor de $[E : F]$. Demuestra que $p(x)$ no tiene ceros en $E$.
    
    \textbf{Solución:}
    
    Si un cero $\alpha$ de $p(x)$ estuviera en $E$, entonces como $p(x)$ es irreducible sobre $F$, tendríamos $[F(\alpha) : F] = \text{deg}(p(x))$, y $[F(\alpha) : F]$ sería un divisor de $[E : F]$ por el Teorema 31.4. Pero por hipótesis, esto no es el caso. Por lo tanto, $p(x)$ no tiene ceros en $E$.
    
    \item Sea $E$ una extensión de campo de $F$. Sea $a$ en $E$ algebraico de grado impar sobre $F$. Demuestra que $a^2$ es algebraico de grado impar sobre $F$, y $F(a) = F(a^2)$.
    
    \textbf{Solución:}
    
    Como $F(a)$ es una extensión finita de $F$ y $a^2 \in F(a)$, el Teorema 31.3 muestra que $a^2$ es algebraico sobre $F$. Si $F(a^2) \neq F(a)$, entonces $F(a)$ sería una extensión de $F(a^2)$ de grado $2$, porque $a$ es una raíz de $x^2 - a^2$. Por el Teorema 31.4, esto significaría que $2$ divide el grado de $F(a)$ sobre $F$, lo cual es imposible ya que el grado de $a$ es impar. Por lo tanto, $F(a) = F(a^2)$.
    
    \item Demuestra que si $F$, $E$ y $K$ son campos con $F \leq E \leq K$, entonces $K$ es algebraico sobre $F$ si y solo si $E$ es algebraico sobre $F$, y $K$ es algebraico sobre $E$. (No debes asumir que las extensiones son finitas.)
    
    \textbf{Solución:}
    
    Supongamos que $K$ es algebraico sobre $F$. Entonces cada elemento de $K$ es una raíz de un polinomio no nulo en $F[x]$, y por lo tanto en $E[x]$. Esto muestra que $K$ es algebraico sobre $E$. Por supuesto, $E$ es algebraico sobre $F$, porque cada elemento de $E$ también es un elemento de $K$.
    
    Recíprocamente, supongamos que $K$ es algebraico sobre $E$ y que $E$ es algebraico sobre $F$. Sea $\alpha \in K$. Debemos mostrar que $\alpha$ es algebraico sobre $F$. Como $K$ es algebraico sobre $E$, $\alpha$ es una raíz de un polinomio no nulo en $E[x]$. Porque $E$ es algebraico sobre $F$, los coeficientes de este polinomio son algebraicos sobre $F$. Por lo tanto, $\alpha$ es algebraico sobre $F$, y $K$ es algebraico sobre $F$.
    
    \item Sea $E$ una extensión de campo de un campo $F$. Demuestra que todo $a$ en $E$ que no está en el cierre algebraico $\overline{F}_E$ de $F$ en $E$ es trascendente sobre $\overline{F}_E$.
    
    \textbf{Solución:}
    
    Si $\alpha$ es algebraico sobre $\overline{F}_E$, entonces $F(\alpha)$ es una extensión finita de $F$, y por lo tanto, $\alpha$ es algebraico sobre $F$. Pero entonces $\alpha$ está en el cierre algebraico de $F$ en $E$, lo cual es una contradicción. Por lo tanto, $\alpha$ es trascendente sobre $\overline{F}_E$.
    
\end{enumerate}


\begin{enumerate}
    \setcounter{enumi}{33}
    
    \item Demuestra que si $E$ es una extensión algebraica de un campo $F$ y contiene todos los ceros en $\overline{F}$ de cada $f(x)$ en $F[x]$, entonces $E$ es un campo algebraicamente cerrado.
    
    \textbf{Solución:}
    
    Sea $\alpha \in E$ y sea $p(x) = \text{irr}(\alpha, F)$ de grado $n$. Ahora, $p(x)$ se factoriza en $(x - \alpha_1)(x - \alpha_2) \cdots (x - \alpha_n)$ en $F[x]$. Debido a que por hipótesis todos los ceros de $p(x)$ en $F$ también están en $E$, vemos que esta misma factorización también es válida en $E[x]$. Por lo tanto,
    \[ p(\alpha) = (\alpha - \alpha_1)(\alpha - \alpha_2) \cdots (\alpha - \alpha_n) = 0, \]
    entonces $\alpha = \alpha_i$ para algún $i$. Esto muestra que $F \leq E \leq \overline{F}$. Debido a que, por definición, $F$ contiene solo elementos que son algebraicos sobre $F$ y $E$ contiene todos estos, vemos que $E = \overline{F}$ y, por lo tanto, es algebraicamente cerrado.
    
    \item Demuestra que ningún campo finito de característica impar es algebraicamente cerrado. (De hecho, tampoco ningún campo finito de característica 2 es algebraicamente cerrado.) [Pista: Mediante un conteo, demuestra que para tal campo finito $F$, algún polinomio $x^2 - a$, para algún $a \in F$, no tiene cero en $F$. Consulta el Ejercicio 32, Sección 29.]
    
    \textbf{Solución:}
    
    Si $F$ es un campo finito de característica impar, entonces $1 \neq -1$ en $F$. Debido a que $1^2 = (-1)^2 = 1$, los cuadrados de los elementos de $F$ pueden recorrer a lo sumo $|F| - 1$ elementos de $F$, por lo que hay algún $a \in F$ que no es un cuadrado. El polinomio $x^2 - a$ entonces no tiene ceros en $F$, por lo que $F$ no es algebraicamente cerrado.
    
\end{enumerate}


