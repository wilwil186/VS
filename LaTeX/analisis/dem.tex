


\textbf{Demostración}: \\

Para demostrar esta identidad, se puede utilizar la definición de $dy_i$ como una aplicación lineal que asigna a cada vector $v \in \mathbb{R}^n$ un escalar $dy_i(v)$.

\begin{equation*}
dy_i(v) = \frac{\partial f_i}{\partial x_1} v_1 + \frac{\partial f_i}{\partial x_2} v_2 + \cdots + \frac{\partial f_i}{\partial x_n} v_n
\end{equation*}

Ahora, se puede expresar $dy_i(v)$ en términos de los elementos básicos del espacio vectorial $dx_j$:

\begin{equation*}
dy_i(v) = \sum_{j=1}^{n} \frac{\partial f_i}{\partial x_j} v_j
\end{equation*}

Finalmente, se puede reescribir esta expresión en términos de los elementos básicos del espacio vectorial $dx_j$:

\begin{equation*}
dy_i = \sum_{j=1}^{n} \frac{\partial f_i}{\partial x_j} dx_j
\end{equation*}

