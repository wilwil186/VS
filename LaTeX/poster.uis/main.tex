\documentclass[final, 20pt, a2paper, portrait]{extarticle}
\usepackage[left=2cm, right=2cm, top=2cm, bottom=2cm]{geometry}

\usepackage[utf8]{inputenc}
\usepackage{xcolor}
\definecolor{nejltblue}{RGB}{63,81,181}


\usepackage{nejlt}
\usepackage{graphicx}
\usepackage{multicol}
\usepackage[spanish]{babel}
\usepackage{titling}
\usepackage{fancyhdr}
\usepackage{background}

\pagestyle{fancy}
\fancyhf{}
\renewcommand{\headrulewidth}{0pt}
\renewcommand{\footrulewidth}{.8pt}
\fancyfoot[R]{\thepage}


\lhead{\includegraphics[width=4cm]{imagenes/Distrital.png}}
\rhead{\includegraphics[width=4cm]{imagenes/Uis.png}}
\chead{Análisis de la Serie de Tiempo de Bitcoin Utilizando Técnicas de  \\ Random Forest y Redes Neuronales Recurrentes LSTM \\ Wilson Eduardo Jerez-Hernández \\\url{wejerezh@udistrital.edu.co} \rule{\textwidth}{1pt}}

% Logo del Semillero IPREA en el fondo del documento
\backgroundsetup{
  scale=1,
  color=black,
  opacity=0.4,
  angle=0,
  position=current page.south,
  vshift=25cm, % Ajusta la posición vertical según sea necesario (valor negativo para ir hacia arriba)
   contents={\includegraphics[width=30cm]{imagenes/Iprea.png}}
}

\title{Análisis de la Serie de Tiempo de Bitcoin Utilizando Técnicas de Random Forest y Redes Neuronales Recurrentes LSTM}
\author{Wilson Eduardo Jerez Hernández\\wejerezh@udistrital.edu.co}
\date{}

\begin{document}
%%%%%%%%%%%%%%%%%%%%%%%%%%%%%
%%truco sucio para que no se superponga el encabezado con el texto
\section*{}
\vspace{2cm}
%%%%%%%%%%%%%%%%%%%%%%%%%%%%%
\section*{Resumen}
En este artículo, se investiga el comportamiento de los precios de Bitcoin a lo largo del tiempo y se exploran dos enfoques diferentes para la predicción: Random Forest y Redes Neuronales Recurrentes con unidades LSTM. El estudio se centra en la volatilidad de las criptomonedas, que ha generado escepticismo debido a su sistema descentralizado y la falta de control por parte de autoridades reguladoras. A medida que las criptomonedas ganan aceptación en el mercado, su alta volatilidad se ha convertido en un tema de preocupación. El objetivo principal de esta investigación es proporcionar una comprensión más profunda de cómo las técnicas de Random Forest y Redes Neuronales Recurrentes pueden aplicarse al análisis de criptomonedas con el fin de reducir esta volatilidad y predecir los valores de las criptomonedas.

\section*{Objetivos}
El objetivo principal de este proyecto es analizar y predecir el comportamiento de los precios de Bitcoin a lo largo del tiempo, abordando la volatilidad característica de las criptomonedas. Para lograr este objetivo, se emplearán dos enfoques diferentes: Random Forest y Redes Neuronales Recurrentes (RNN) con unidades LSTM.

\section*{Fundamentos Teóricos}
En esta sección, se presentarán los conceptos fundamentales relacionados con el análisis de series de tiempo y el uso de algoritmos de aprendizaje automático, incluyendo árboles de decisión, Bagging, Bosques Aleatorios y Deep Learning.

\section*{Metodología}
Las metas específicas son las siguientes:
    \begin{enumerate}
        \item Investigar y adquirir conocimientos sobre el algoritmo Random Forest, las Redes Neuronales Recurrentes y su aplicación en el análisis de criptomonedas.
        \item Recopilar datos históricos del precio de Bitcoin desde el 31 de diciembre de 2022 hasta el 30 de junio de 2023, obtenidos de Yahoo Finance.
        \item Construir modelos predictivos utilizando el algoritmo Random Forest y Redes Neuronales Recurrentes con unidades LSTM para predecir los precios futuros de Bitcoin.
    \end{enumerate}

\section*{Resultados}
Los resultados obtenidos por los dos enfoques serán comparados y contrastados, identificando sus fortalezas y debilidades en la predicción de los precios de Bitcoin.
%%%%%%%%%%%%%%%%%%%%%%%%%%%%%
%%truco sucio para que no se superponga el encabezado con el texto
\section*{}
\vspace{2cm}
%%%%%%%%%%%%%%%%%%%%%%%%%%%%%
\section*{Conclusiones}
Se elaborarán conclusiones basadas en los resultados y la literatura relevante, ofreciendo una visión integral de la efectividad de Random Forest y Redes Neuronales Recurrentes en la reducción de la volatilidad y la precisión de las predicciones de las criptomonedas.

\begin{thebibliography}{9}

\bibitem{carrion}
Carrión Pérez, Rafael. (2020). \textit{Predicción de Precios de Criptomonedas con Random Forest}, \url{https://oa.upm.es/62842/}, UPM.

\bibitem{mdpi}
Chen, Junwei. (2023). \textit{Analysis of Bitcoin Price Prediction Using Machine Learning}. \textit{MPDI}, \url{https://www.mdpi.com/1911-8074/16/1/51}

\bibitem{towardsdatascience}
Abhijit, Roy. (2020). \textit{A Dive Into Decision Trees}. \textit{towardsdatascience}, \url{https://towardsdatascience.com/a-dive-into-decision-trees-a128923c9298}

\bibitem{jrfm}
Journal of Risk and Financial Management. MDPI. \url{https://www.mdpi.com/journal/jrfm}
\bibitem{yahoo}
Yahoo! en español. \url{https://espanol.yahoo.com/}.

\bibitem{nakamoto}
Nakamoto, Satoshi. 2008. \textit{Bitcoin: un sistema de efectivo electrónico entre pares}. Disponible en línea: \url{https://bitcoin.org/bitcoin.pdf}

%For a book:
%\bibitem{author_year} Author. (Year). \textit{Title of Book}. Publisher.
%For a journal article:
%\bibitem{author_year} Author. (Year). Title of Article. \textit{Name of Journal}, \textit{Volume}(Issue), page numbers.
%For a website:
%\bibitem{author_year} Author. (Year). Title of Webpage. \textit{Name of Website}. URL

\end{thebibliography}

\end{document}
